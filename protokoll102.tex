\documentclass[11pt,ngerman,a4paper]{article}
%Gummi|061|=)
\usepackage{amsmath}
\usepackage{a4wide}
\usepackage{amsthm}
\usepackage{amsbsy}
\usepackage{amssymb}
\usepackage{inputenc}
\usepackage{rotating} 
\usepackage{graphicx}
\usepackage{paralist}
\usepackage{selinput}
\SelectInputMappings{%
adieresis={ä},
germandbls={ß},
}
\title{\textbf{Versuch V102: Drehschwingungen}}
\author{Martin Bieker\\
		Julian Surmann\\
		\\
		Durchgef\"{u}hrt am 23.01.2013\\
		Tu Dortmund}
\date{}
\usepackage{graphicx}
\begin{document}
\renewcommand\tablename{Tabelle}
\renewcommand\figurename{Abbildung}
\maketitle
\thispagestyle{empty}
\newpage
\clearpage
\setcounter{page}{1}

\section{Einleitung}
Der folgende Versuch behandelt Drehschwingungen. Dabei wird ein Metalldraht, an dem eine Masse hängt, ausgelenkt und so in periodische Schwingungen versetzt.Im ersten Versuchsteil werden die elastischen Konstanten untersucht. Bei diesen handelt es sich um Proportionalitätsfaktoren zwischen den relativen Verformungen und den pro Fläche angreifenden Flächen.
Im zweiten Versuchsteil wird (auch mit Drehschwingungen) das magnetische Moment eines Permanentmagneten in der Kugel bestimmt.
\section{Theorie}
In der Physik gibt es zwei Arten von Kräften, die auf Festkörper wirken: Entweder, die Kräfte wirken auf jedes Volumenelement (z.B. die Schwerkraft), oder sie wirken nur auf die Körperoberfläche. Erstere sind in der Lage, eine Änderung des Bewegungszustandes zu bewirken. Letztere können nur Volumens- oder Gestaltsänderungen erzeugen und spielen in diesem Versuch die entscheidende Rolle. Die sogenannte Spannung beschreibt die Oberflächenkraft pro Fläche. Diese kann in eine Normalspannung und eine Tangentialspannung aufgeteilt werden. Ist die Spannung hinreichend klein, sind die auftretenden Längen- oder Volumenänderungen proportional zu diesen Anteilen:
\begin{equation}
\label{1}
\sigma = E\frac{\Delta L}{L}
\end{equation}
\begin{equation}
\label{2}
P = Q\frac{\Delta V}{V}.
\end{equation}
Dieser Zusammenhang heißt Hook'sches Gesetz.\newline
In diesem Versuch wird das elastische Verhalten von isotropen Körpern betrachtet. Das heisst, dass die elastischen Konstanten richtungsunabhängig sind. Zwei Konstanten reichen dann aus, um das elastische Verhalten vollständig zu beschreiben. Es existieren viel elastische Konstanten:
\begin{itemize}
\item Torsionsmodul G: Beschreibt die Gestaltselastizität.
\item Kompressionsmodul Q: Beschreibt die Volumenelastizität.
\end{itemize}

\section{Aufbau und Durchf\"{u}hrung}

\section{Auswertung}

\begin{figure}[h!]
\centering
%\includegraphics[scale=0.7]{Abb/abb2.png}
\caption{Halblogarithmische Darstellung von Abb \ref{abb1}}
\label{abb2}
\end{figure}

\section{Diskussion}

\section{Literatur- und Abbildungsverzeichnis}
\begin{itemize}
\item $[1]$: Der Praktikumsanleitung zu V354 der TU Dortmund entnommen. Download am 5.1.14 unter \newline http://129.217.224.2/HOMEPAGE/PHYSIKER/BACHELOR/AP/SKRIPT/V354.pdf
\end{itemize}
\section{Anhang}
\begin{itemize}
\item Tabellen und Abbildungen
\item Auszug aus dem Messheft.


\end{itemize}

\newpage
\begin{table}[H]
\centering
\begin{tabular}{|c|c|c|}
\hline
$\frac{t}{ms}$ & $\frac{U_C}{V}$ & $ln(\frac{U_C}{V})$ \\
\hline
0.0046 & 18.6 & 2.923\\
0.0354 & 15.6 & 2.747\\
0.0642 & 13.2 & 2.58\\
0.0938 & 11.2 & 2.416\\
0.1234 & 9.4 & 2.241\\
0.1516 & 7.8 & 2.054\\
0.1812 & 6.6 & 1.887\\
0.211 & 5.6 & 1.723\\
0.2404 & 4.6 & 1.526\\
0.2704 & 4.0 & 1.386\\
0.3004 & 3.4 & 1.224\\
0.329 & 2.8 & 1.03\\
0.3578 & 2.2 & 0.788\\
0.3874 & 2.0 & 0.693\\
0.4184 & 1.8 & 0.588\\
0.4478 & 1.4 & 0.336\\
\hline
\end{tabular}
\label{mtab1}
\caption{Lokale Maxima des Spannungsverlaufs}
\end{table}

\end{document}
